\subsection{Redes Neuronales}
Las redes neuronales son estructuras de predicción de propósito general, que se basan en unidades, llamadas \emph{neuronas}. Particularmente en este caso, vamos a usar una red neuronal para clasificación, y veremos como influye eso a la hora de construirla.

Históricamente %Buscar referencia!
surgieron de un acercamiento al aprendizaje automático que seguía como heurística conseguir un parecido entre la inteligencia artificial y el funcionamiento de la mente humana. Una red tiene capas de neuronas, cada una de las cuales obtiene con input una combinación lineal de los datos de la capa anterior. Cada neurona, a ese input, le aplica una \emph{función de activación}, que será su output.

Para el caso de una sola neurona de clasificación, podemos comparar el comportamiento con el de una regresión logística:

\begin{figure}[h]
\centering
\def\svgwidth{0.25\columnwidth}
\caption{Neurona clasificadora}
\input{./images/neurona.pdf_tex}
\end{figure}

En la figura se ve una capa $l$ (de "layer") con tres neuronas. Cada una tiene como output su función de activación $a$. La neurona de la capa siguiente toma como input una combinación lineal de las anteriores y le aplica su propia función de activación, la función sigmoidea.
\begin{eqnarray}
 h_{\theta} = g(z)  \\
 z = \sum_{i = 1}^{3}\theta_i a_i^{(l)}=\theta^T a^{(l)}
\end{eqnarray}
De esta forma obtenemos la ya vista regresión logística.

Una red neuronal es más general. Tiene una capa inicial, con los datos de entrada para la predicción, una capa de salida, con la hipótesis, y una cantidad arbitraria de capas llamadas \emph{ocultas} entre ellas. Estas últimas permiten que la hipótesis final sirva para modelar problemas muy complejos, ya que las activaciones de cada capa anterior actúan como nuevas features para las capas siguientes.