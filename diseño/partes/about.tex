\section{Acerca del equipo} % (fold)
\label{sec:acerca_de}

\subsection{Filosofía de trabajo} % (fold)
\label{sub:filosofia}

Nuestra filosofía para encarar esta competencia fue primeramente reunir toda la información y el conocimiento que creimos necesario para entender el problema. Para esbozar una solución y desarrollar un algoritmo de aprendizaje que consiga hacer una clasificación óptima de los crímenes de San Francisco decidimos primero entender el problema en sus distintas aristas. Esto implicó entre otras cosas:

\begin{itemize}
  \item Revisar los comentarios hechos hasta el momento en los foros de Kaggle. La competencia lleva ya varios meses de desarrollo, y es una buena forma de hacer un primer acercamiento aprender de los comentarios de los participantes que ya comenzaron a desarrollar su solución.
  \item Investigar acerca del dominio del problema planteado en la competencia. Las características geográficas de la ciudad de San Francisco, el grado de importancia de los features incluidos en el set de entrenamiento y el set de evaluación. Análisis y comprensión del significado de los features. Por ejemplo, entender qué tipo de eventos comprende la categoría de crimen ``Non criminal''. 
  \item Compartir los conocimientos previos de los integrantes del equipo en el campo de Data Science, más específicamente en el área de machine learning. Complementar estos conocimientos previos con nuevas investigaciones que nos permitan aprender el funcionamiento de los algoritmos de machine learning existentes, y aún más importante, la idoneidad de los mismos para abordar este problema.
\end{itemize}

\subsection{Plan de trabajo} % (fold)
\label{sub:plan}

Desde la conformación del equipo nos dedicamos primeramente a hacer investigaciones con la extensa información que hay en internet, ya sea para aprender los algoritmos, como el dominio del problema y la matemática involucrada.

A continuación, una vez aprendidas las nociones básicas de varios algoritmos y soluciones posibles, pasamos a investigar sobre posibles librerías que nos faciliten el tratamiento de los features de los sets de entrenamiento y evaluación, así como la implementación de algunos de los algoritmos investigados. Así fue como encontramos comodidad en el entorno que ofrece python para rápidamente procesar los datos de entrada y simular, a través de librerías como el Sci-kit, los algoritmos que más nos llamaron la atención y nos parecieron útiles para desarrollar una solución.

Simultáneamente evaluamos y propusimos diversas formas de generar nuevos features a partir de los que aporta el set de entrenamiento. Consideramos que la forma de representar la información otorgada por la competencia, y de alimentar los algoritmos es clave para lograr un aprendizaje más eficiente en ellos.

Finalmente comenzamos a pensar en la implementación de los algoritmos que más nos gustaron en C++, y las librerías que consideramos útiles para facilitarlo. De aquí en adelante resta comenzar a desarrollar la implementación, y dejar abierta la posibilidad de combinar soluciones para obtener mejores resultados.