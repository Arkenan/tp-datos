\section{Implementación en C++}
Para que la implementación de los algoritmos sea eficiente, la mayor parte de los resultados a utilizar fueron expresados en forma matricial, de forma de poder aprovechar las alta performance en estas operaciones de bibliotecas como \textit{Armadillo}. Esta última, que es la que estamos usando en este momento, tiene además un tipo de datos SpMat especialmente destinado al trabajo con matrices dispersas, lo cual puede ser de particular utilidad por el acercamiento que tomamos respecto a las features discretas.

Desde el punto de vista de diseño, tendriamos que dividir la solución en pasos del flujo de datos, desde la entrada del archivo csv inicial, hasta el archivo csv con las probabilidades calculadas, pasando por la limpieza y conversión de datos, aprendizaje y predicción de los algoritmos, posible promedio entre las salidas de los algoritmos. La ídea es hacer que esos pasos sean separados e independientes, facilitandonos de esta manera las mediciones de tiempos y búsqueda de errores. Idealmente la salida de cada paso tendría que poder ser guardada, para luego seguir desde ese punto. Probablemente necesitemos alguna libreria que pueda parsear eficientemente un csv de entrada y serializar los datos en forma de csv tambien.
