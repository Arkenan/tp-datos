\section{Introducción} % (fold)
\label{sec:introducci_n}

El trabajo práctico consiste en proponer una solución a la consigna de una competencia abierta en Kaggle, una plataforma de la comunidad de Data Science. La competencia consiste en la predicción de los crímenes ocurridos en San Fransisco a partir de los datos proporcionados. Estos datos se dividen en un \textit{set de entrenamiento}, un archivo csv, donde cada fila representa un crímen y cada columna proporciona información acerca de él, y un \textit{set de prueba}. Este último tiene el mismo formato, pero en las filas no se menciona de que crímen se está tratando. Queda a cargo de los participantes el deducir la probabilidad de la ocurrencia de cada \textit{clase} de crímen para cada fila del set de prueba.


Nosotros vamos a encarar el problema de la predicción de las clases de crímen utilizando varios algoritmos de \textit{machine learning}: \textbf{Regresión Logística}, \textbf{Red Neuronal} (y analizaremos otros algoritmos posibles). Dependiendo de la calidad de la solución de cada algoritmo, juntaremos los mejores \textit{modelos} para tener un resultado más preciso. Necesitaremos procesar y preparar los datos, extraer los diferentes atributos (o \textit{features}) del set de datos para que los algoritmos se entrenen de manera más eficiente. La solución final va a estar escrita en \textbf{C++}, pero realizaremos pruebas en lenguajes de más alto nivel (\textbf{Python} mayormente), aprovechando las herramientas de ML y procesamiento de datos existentes.

% section introducci_n (end)
